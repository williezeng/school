\documentclass[../report.tex]{subfiles}
\begin{document}

    \begin{frame}
        \frametitle{3a: Replacement of Pixels}
        \begin{figure}[!htb]
            \centering
            \frame{\includegraphics[keepaspectratio,height=0.65\textheight,width=0.45\textwidth]{ps1-3-a-1}}
            \caption{ps1-3-a-1}
        \end{figure}
    \end{frame}
    
\end{document}